\documentclass[10pt,a4paper]{article}
\usepackage[utf8]{inputenc}
\usepackage[italian]{babel}
\usepackage{amsmath}
\usepackage{amsfonts}
\usepackage{amssymb}
\usepackage{graphicx}
\usepackage[left=2cm,right=2cm,top=2cm,bottom=2cm]{geometry}
\newcommand{\rem}[1]{[\emph{#1}]}

\author{Gruppo AC \\ Federico Belliardo, Giulia Franchi, Francesco Mazzoncini}
\title{Esercitazione N.4: Amplificatore a transistor}
\begin{document}

\maketitle

\section{Scopo dell'esperienza}
L'esercitazione ha come scopo quello di realizzare un circuito amplificatore, utilizzando un transistor \textit{npn} 2N1711.

\section{Montaggio del circuito e verifica del punto di lavoro}
Abbiamo montato il circuito in Fig.1 come richiesto, con: $R_1= 179\pm1  k\Omega$, $R_2= 18.0\pm0.1  k\Omega$, $R_C= 9.95\pm 0.09  k\Omega$, $R_E= 0.987\pm 0.008 \, k\Omega$, $C_{IN}= 230\pm10  nF$, $C_{OUT}= 99\pm4  nF$ e $C_E= 100\pm20  \mu F$. Tutti i componenti sono stati misurati con multimetro digitale, tranne il condensatore elettrolitico di cui abbiamo assunto il valore nominale.\\
Supponiamo che il transistor lavori in zona attiva, $V_{BE} = 0.6 \pm 0.1 \, V$. Abbiamo inoltre definito $V_{PART}=V_{CC}\frac{R_2}{R_1+R_2}$, per determinarla abbiamo prima misurato la tensione in ingresso $V_{CC}= 20.2\pm0.1 V$, così la tensione ai capi del partitore risulta $V_{PART}=1.85 \pm 0.01\,V$.


\subsection{Misura del punto di lavoro}
Per la determinazione del punto di lavoro del circuito si è misurato $V_{CE}= 7.50\pm0.04 \, V$ e la caduta di potenziale ai capi della resistenza $R_C$, $V_{R_C}= 11.62\pm0.05  V$, in modo da poter determinare $I_C=\frac{V_{RC}}{I_C}= (1.17\pm0.01)\,mA$.
Confrontando questi valori misurati con i valori teorici, $I_C=\frac{V_{PART}-V_{BE}}{R_E}= 1.12\pm0.01\,mA$ e $V_{CE}=V_{CC}-(R_C+R_E)I_C= 7.4\pm0.1\,V$, si vede che si ha compatibilità entro l'errore.\\
La retta di lavoro attesa è  $V_{CC}=V_{CE}+I_C(R_C+R_E)$.


\subsection{Misura delle tensioni ai terminali del transistor}
Abbiamo misurato le tensioni $V_B= 1.77\pm0.01\,V$, $V_E= 1.17\pm0.01\,V$, $V_{BE}= 0.605\pm0.004\,V$ e $V_C= 8.47\pm0.04\,V$. Sono compatibili con quanto abbiamo calcolato teoricamente?! $V_B= V_{PART}=1.85 \pm 0.01\,V$, $V_E=I_C^Q R_E = 1.2 \pm 0.1 \,V$, $V_C=V_{CC}-I_C^Q R_C = 8 \pm 1 \, V$.


\subsection{Valutazione della corrente di base}
\rem{Manca!}
Ci aspetteremmo una corrente di base $I_B=\frac{I_C}{h_{FE}}= \mu A$, dato che si suppone il transistor lavori in zona attiva. Misuriamo le cadute di potenziale ai capi delle resistenze $R_1$ e $R_2$, $V_{R1}= \pm V$ e $V_{R2}= \pm V$, da cui abbiamo ricavato $I_{R1}= \pm \mu A$ e $I_{R2}= \pm \mu A$, dalle quali infine abbiamo ricavato $I_B=I_{R1}-I_{R2}= \pm \mu A$. 

\section{Risposta a segnali sinusoidali di frequenza fissa}
In questa parte dell'esperienza si è utilizzato un segnale ad una frequenza fissa pari a $f= 5.00\pm0.05 Hz$.

\subsection{Misura del guadagno in tensione}
Abbiamo preso diverse misure di $V_{OUT}$ (è dello sfasamento di tempo rispetto all'ingresso) in funzione di $V_{IN}$ al variare di quest'ultimo, prestando attenzione ai fenomeni di clipping. Nella tabella \ref{ampiezza} seguenze riportiamo le nostre misure, aggiungendo anche il calcolo del guadagno in tensione, $A_V=\frac{V_{OUT}}{V_{IN}}$ e dello sfasamento angolare rispetto a $\pi \,rad$.\\
L'oscilloscopio è stato utilizzato in modalità AC.\\
Tutti i dati mostrati nelle tabelle e nei grafici (in tutta la relazione) riportano sia l'errore sistematico che che quello statistico sommati in quadratura, in tutti i fit e le propagazioni sono stati considerati gli errori statistici.

\begin{table}[!hbt]
\centering
\begin{tabular}{|c|c|c|c|c|c|c|c|}
\hline 
$V_{IN}$ [V] & $\sigma V_{IN}$ [V] & $V_{OUT}$ [V]& $\sigma V_{OUT}$ [V] & $\phi - \pi$ [rad] & $\sigma \phi$ [rad] & $A_V$ & $\sigma A_V$ \\ 
\hline
0.206 & 0.006 & 2.00 & 0.06 & 0.06 & 0.07 & 9.7 & 0.1\\
0.294 & 0.009 & 2.86 & 0.09 & 0.06 & 0.07 & 9.73 & 0.08\\
0.42 & 0.01 & 2.00 & 0.06 & 0.09 & 0.07 & 4.76 & 0.05\\
0.51 & 0.02 & 5.0 & 0.2 & 0.06 & 0.07 & 9.84 & 0.05\\
0.62 & 0.02 & 6.0 & 0.2 & 0.03 & 0.07 & 9.61 & 0.04\\
0.71 & 0.02 & 6.9 & 0.2 & 0.06 & 0.07 & 9.66 & 0.04\\
0.80 & 0.02 & 7.7 & 0.2 & 0.06 & 0.07 & 9.60 & 0.05\\
0.90 & 0.03 & 8.7 & 0.3 & 0.09 & 0.07 & 9.73 & 0.06\\
1.03 & 0.03 & 9.9 & 0.3 & 0.06 & 0.07 & 9.63 & 0.05\\
1.12 & 0.03 & 10.6 & 0.3 & 0.03 & 0.07 & 9.46 & 0.05\\
1.21 & 0.04 & 11.5 & 0.3 & 0.09 & 0.07 & 9.50 & 0.05\\
1.30 & 0.04 & 12.6 & 0.4 & 0.06 & 0.07 & 9.69 & 0.04\\
\hline
\end{tabular}
\caption{Misure di tensione, guadagno e fase.} \label{ampiezza}
\end{table}
Si può vedere come il segnale in uscita sia sfasato di $\pi \, rad$ come atteso dai calcoli teorici. Tutte le misure dello sfasamento sono compatibili con zero entro l'errore sperimentale. Tuttavia si può osservare una fase sistematicamente maggiore di $\pi \, rad$ probabilmente a causa dell'impedenza dei condensatori in ingresso e uscita. Il valore medio dell'attenuzione (con errore propagato in maniera statistica sulla media di tutte le misure) è: $A = (-9.24 \pm 0.02)$.

La tensione misurata a cui inizia il \emph{clipping} inferiore è circa $V_{inf} = 1.4 \,V$, mentre il clipping superiore inizia circa a $V_{sup} = 2.2 \, V$.\\
Questo è dovuto al fatto che il punto di quiescenza scelto è più vicino alla zona di saturazione che alla zona di interdizione. Il clipping inferiore corrisponde alla zona di saturazione, perchè significa che $V_{IN}$ ha il valore massimo e quindi la corrente di base è tale da mandare il transistor in saturazione.\\
Quando si ha clipping superiore la $V_{IN}$ è al minimo valore dunque non ho polarizzazione della base e sono in interdizione.
Entrambi gli effetti accennati sono effetti non lineari del transistor, cioè deviazioni dal comportamento ideale in cui vengono mandati segnale armonici in segnali armonici.\\


\subsection{Impedenza di ingresso del circuito}
Come impedenza in ingresso del circuito ci aspettiamo $R_{IN}=R_1//R_2//(h_{ie}+h_{fe}R_E) = 15.3\pm0.1$.
Abbiamo misurato la tensione in uscita del circuito in Fig \ref{circuito}: $V_{OUT,1} = (6.44\pm0.02) \, V$, e successivamente abbiamo inserito una resistenza, $R_S= (18.1\pm0.1) \, k\Omega$, fra il generatore e $C_{IN}$, misurando poi $V_{OUT,2}= (2.94 \pm 0.02) \, V$. Il valore di $V_{IN} = (668 \pm 2) \, mV$, che rimane costante durate le due misure. Utilizzando la formula $\frac{R_S}{R_{IN}}=\frac{V_{OUT,1}}{V_{OUT,2}}-1$ ci è stato possibile ricavare il valore dell'impedenza in ingresso, $R_{IN}= (15.2\pm0.2)\, k\Omega$. 
Per il calolo della resistenza di ingresso teorica la resistenza dinamica della giunzione è stata stimata dalla misure prese assumendo il coefficiente $h_fe = 170\pm10$ determinato nella scorsa esperienza. Si ottiene un accordo ottimo della misura e della stima teorica

\subsection{Impedenza di uscita del circuito}
Come impedenza di uscita del circuito ci aspettiamo $R_{OUT}= R_C = 9.95\pm0.09\, k\Omega$. Come in precedenza abbiamo effettuato due misure di tensione: la prima con il circuito di partenza, $V_{OUT,1}= (6.40\pm0.02) \, V$, la seconda è stata presa misurata dopo aver inserito tra l'uscita e la massa una resistenza di carico $R_L = (9.71 \pm 0.08) \, k\Omega$, $V_{OUT,2}= (3.62\pm0.02)\,V$. La tensione in ingresso è: $V_{IN} = 664 \pm 2) \,$. Grazie alla formula $\frac{R_{OUT}}{R_{L}}=\frac{V_{OUT,1}}{V_{OUT,2}}-1$ abbiamo ottenuto $R_{OUT}= (7.5\pm0.1)\, k\Omega$. La discrpanza può essere imputata a all'impedenza del condensatore $C_{OUT}$ che stata trascurata.


\section{Risposta in frequenza}
\rem{inserire quanto vale la tensione in ingresso costante}
Abbiamo misurato la risposta in frequenza del circuito variando la frequenza da 10 Hz a 1MHz, con una tensione in ingresso $V_{IN,pp}= \pm V$. Quest'ultima è stata controllata più volte durante l'esperienza per far si che rimanesse costante durante la presa dati. Nella tabella sottostante sono riportate le misure effettuate.

le misure effettuate le abbiamo poi riportate in un diagramma di Bode e abbiamo eseguito un fit a tre rette del diagramma per determinare la frequenze di taglio superiori e inferiori. Di seguito sono riportate i arametri delle tre rette e le frequenze determinate.

Si sono effettuate misure delle frequenze di taglio bassa e alta, misuarando le frequenze per cui il segnale è ridotto di un fatto $0.707$.

\section{Aumento del guadagno}
In quest aultima parte si è inserita la resistenza $R_{es}= \pm k\Omega$ e si è misurato il nuovo guadagno a frequenza fissa, $f= \pm Hz$, utilizzando lo stesso metodo e la stessa formula sopra citati. I valori ottenuti sono riportati nella seguente tabella.

\begin{table}[h]
\centering
\begin{tabular}{|c|c|c|c|c|c|}
\hline 
$V_{IN}$ & $\sigma V_{IN}$ & $V_{OUT}$ & $\sigma V_{OUT}$ & $A_V$ & $\sigma A_V$ \\ 
\hline 
• & • & • & • & • & • \\ 
\hline 
• & • & • & • & • & • \\ 
\hline 
• & • & • & • & • & • \\ 
\hline 
• & • & • & • & • & • \\ 
\hline 
• & • & • & • & • & • \\ 
\hline 
• & • & • & • & • & • \\ 
\hline 
• & • & • & • & • & • \\ 
\hline 
• & • & • & • & • & • \\ 
\hline 
• & • & • & • & • & • \\ 
\hline 
\end{tabular}
\caption{Guadagno per piccoli segnali?!.}
\end{table}

Il guadagno atteso per piccoli segnali è $A_V=-\frac{R_C}{Z_E+h_{ie}/h_{fe}}\approx\frac{R_C}{Z_E}$. Confronto con quello misurato.

\end{document}


